\documentclass{article} % For LaTeX2e
\usepackage{nips14submit_e,times}
\usepackage{hyperref}
\usepackage{url}
\usepackage{amsmath,amsfonts,amsthm}
\usepackage{bbm}
\usepackage{algorithm,algorithmic}
\usepackage{graphicx}
\usepackage{bm}
\usepackage{bbm}
\usepackage[titletoc]{appendix}
\usepackage{wrapfig}
\usepackage{afterpage}
\usepackage{amssymb}
\usepackage{booktabs}
\usepackage{ulem}
\usepackage{multirow}

\def\B#1{\bm{#1}}
%\def\B#1{\mathbf{#1}}
\def\trans{\mathsf{T}}

%\renewcommand{\labelitemi}{--}

\newtheorem{theorem}{Theorem} \newtheorem{lemma}[theorem]{Lemma}
\newtheorem{proposition}[theorem]{Proposition}
\newtheorem{corollary}[theorem]{Corollary}
\newtheorem{definition}[theorem]{Definition}
\newtheorem{remark}{Remark}

%%%%%%%%%%%%%%%%%%%%%%%%%%%%%%%%%%%%%%%%%%%%%%%%%%%%%%%%%%%%%%%%%%%%%%%%%%%%%%%

\title{The Default Mode of Brain Function\\ Viewed By Control Theory}

\newcommand{\fix}{\marginpar{FIX}}
\newcommand{\new}{\marginpar{NEW}}
\DeclareMathOperator{\proj}{proj}
\DeclareMathOperator{\softmax}{softmax}
\DeclareMathOperator{\prox}{prox}
\DeclareMathOperator{\Prox}{Prox}
\DeclareMathOperator{\im}{im}

% macros from michael's .tex
\DeclareMathOperator{\dist}{dist} % The distance.
\DeclareMathOperator{\argmin}{argmin}
\DeclareMathOperator{\argmax}{argmax}
\DeclareMathOperator{\Id}{Id}
\DeclareMathOperator{\abs}{abs}
\newcommand{\R}{\mathbb{R}}
\newcommand{\N}{\mathbb{N}}
\newtheorem{thm}{Theorem}[section]
\newtheorem{prop}[thm]{Proposition}
\newtheorem{lem}[thm]{Lemma}
\newtheorem{cor}[thm]{Corollary}

\def\Id{\mathbf{I}}
\def\1{\mathbf{1}}
\def\X{\mathbf{X}}
\def\U{\mathbf{U}}
\def\V{\mathbf{V}}
\def\v{\mathbf{v}}
\def\u{\mathbf{u}}
\def\z{\mathbf{z}}
\def\Y{\mathbf{Y}}
\def\A{\mathbf{A}}
\def\B{\mathbf{B}}
\def\C{\mathbf{C}}
\def\N{\mathbf{N}}
\def\R{\mathbf{R}}
\def\Q{\mathbf{Q}}
\def\P{\mathbf{P}}
\def\K{\mathbf{K}}
\def\a{\mathbf{a}}
\def\b{\mathbf{b}}
\def\s{\mathbf{s}}
\def\x{\mathbf{x}}

\newcommand{\suggestadd}[1]{{\color{blue} #1}}
\newcommand{\suggestremove}[1]{{\color{red} \sout{#1}}}

% \nipsfinalcopy % Uncomment for camera-ready version
\nipsfinaltrue
%%%%%%%%%%%%%%%%%%%%%%%%%%%%%%%%%%%%%%%%%%%%%%%%%%%%%%%%%%%%%%%%%%%%%%%%%%%%%%%

\begin{document}

\author{Elvis Dohmatob, (Bert Kappen), Danilo Bzdok\\
  INRIA, Parietal team, Saclay, France\\
  CEA, Neurospin, Gif-sur-Yvette, France\\
  firstname.lastname@inria.fr}

\maketitle

\begin{abstract}
We use linear time-varying dynamic control theory to study the infamous (human)  Default Mode Network.
%

% OUR keywords
%\textbf{\\keywords}: curse of dimensionality; semi-supervised learning;
%fMRI; systems neuroscience

% official NIPS keywords
\textbf{\\keywords}: Systems Biology, Cognitive Science, Autonomous Learning Systems

\end{abstract}

\section{Introduction}
%
\paragraph{The Human Default Mode Network.}
\begin{itemize}
  \item includes brain regions with highest baseline energy consumption in humans
  \item most active in non-disturbed mind-wandering without external influence
  \item tends to deactivate when humans engage in goal-direct tasks
  \item fluctuations in the DMN are related to lapses during and performance in
  such externally structured/focused tasks
  \item includes brain regions that come online first when waking up from
  anesthesia
  \item its network nodes are late to myelinate, an indicator of cognitive sophistication
  \item includes the most advanced processes hierarchies and has
  no direct connections with sensory regions processing external input
  \item the DMN's conceivable key role in the continuous environmental
  tracking in a generative, integrative process might explain both its highest energy consumption in the brain and its intimate coupling with conscious awareness
  \item Indeed, patients with right IPL (=DMN) damage have particular difficulties
  with multistory actions (Hartmann et al. 2005)
  and with imagined action (Sirigu et al. 1996).
  \item Lesion of the hippocampus, feeding memory and spatial information, to the
  DMN impairs future and hypothetical thinking (cf. Hassabis PNAS)
  \item a lesion study on disturbed sleep (i.e., a state of mind
  independent of sensory stimulation but dependent on
  internally generated information) exclusively identified
  the dmPFC (Koenigs et al., 2010). Third, another VLSM study exclusively
  related the IFG and TPJ, both more strongly connected to the
  dmPFC in our study, to inner speech
  \item Bálint's syndrome: neurological disorder of conscious
  awareness resulting from damage in bilateral parietal cortex;
  "psychic paralysis of gaze", that is, the inability for internally
  motivated, targeted saccades hindering fixation of currently not
  attended features in the field of view (i.e., oculomotor apraxia).
  unable to bind various individual features of the visual
  environment into an integrated whole (i.e., simultanagnosia)
  as well as unable to navigate hand movement to a targeted
  object helped by vision (i.e., optic ataxia).
  difficulty avoiding running into objects placed ahead;
  challenged in estimating distances as well as perceiving
  spatial depth, object sizes and object orientation
\end{itemize}

\paragraph{The Predictive Coding Framework.}
\begin{itemize}
  \item It is one of the most frequently cited hypotheses of default mode function
  \item Within this model the brain is conceptualized as a Bayesian machine
  \item Predictive coding is a framework that, in a hierarchical setting,
  is equivalent to empirical Bayesian inference
  \item External sensory input is feed-forward/bottom-up processed in the brain
  \item Their processing is compared against expected input
  \item The prediction error, i.e. the difference between sensory input
  and internal prediction, is computed at each level and passed to
  higher levels via forward connections
  \item in case of a mismatch, back-projections top-down modulate
  input processing
  \item ‘contextual integration’ for top-down modulation of sensorimotor
  processing by context-specific a priori information.
  \item Closely related to Friston's free-energy principle: brain as inference engine
  biological systems, including brains, must minimize the long-term average of surprise;
  formulate perception as a constructive process based on internal or generative models
  \item A generative model of the world
\end{itemize}



\paragraph{Control Theory.}
\begin{itemize}
\item mathematically formalized dynamics systems with input, output and
feedback loops that produce error signals for adaptation
\item typically realized by (non-linear) differential equations
\item has been used to describe neural systems in the past
\item gradient update = learn/adapt system dynamics by error occurrence
\item related to notions of game theory
\end{itemize}



\section{Methods}
\subsection{Linear control network model for brain organization}
Consider the following linear time-invariant (LTI) dynamical system as a (toy) model for high-level brain function
\begin{equation}
  \dot{\x}(t) = \A\x(t) + \B\u(t).
  \label{eq:lti}
\end{equation}
Here, the $n$-by-$n$ matrix $\A$ denotes a model of the brain's wiring (for example a resting state connectome computed from an anatomical atlas), while the $n$-by-$k$ ``input matrix'' $\B$ describes which nodes can be controlled by us, an external \textit{controller}, via medical intervention or a careful choice of stimulus presentation, for example. 
At time $t \ge 0$, let  the $n$-vector  $\x(t) := (\x_1(t),\ldots,\x_n(t))$ encodes the state of the network (one value for each node). The aim is to supply values of $\u_1(t),\ldots,\u_k(t)$ for $k \le n$ controls as a function of the time $t$,  to take the system from any prescribed initial state $\x(0) = \x^{\text{init}} \in \mathbb R^n$ to any prescribed final state $\x(\tau) = \x^{\text{fin}} \in \mathbb R^n$ in finite time $\tau < \infty$.
When such a controlling is possible, we say that the system $(\A,\B)$ is \textit{controllable}. A precise sufficient and necessary condition for such controllability is the Kalman condition: $(\A,\B)$ is controllable iff the $n \times nk$ \textit{controllability matrix}
\begin{equation}
  \C := (\B|\A\B|\ldots|\A^{n-1}\B)
\end{equation}
has full rank, i.e
\begin{equation}
  rank(\C) = n.
\end{equation}

\paragraph{How many controls do we need at best ?}
In the thermodynamic limit ($n \rightarrow \infty$), statistical physics \cite{Liu2011} gives the extremely good estimate
\begin{equation}
  n_0(\langle k\rangle, \gamma) \approx e^{-\frac{1}{2}\left(\frac{\gamma-2}{\gamma - 1}\right)\langle k \rangle},
\end{equation}
where $\gamma \ge 1$ is the \textit{scale-free} parameter for the node-degree distribution of the network described by $\A$, and $\langle k \rangle$ is the mean degree. Letting $\gamma \rightarrow \infty$, one recovers the Erdos-Renyi value
\begin{equation}
  n_0(\langle k\rangle) = e^{-\frac{1}{2}\langle k\rangle},
\end{equation}
which agrees perfectly with its known analytic value. One identifies two radically different regimes:
\begin{itemize}
  \item the manageable regime $\gamma > \gamma_c := 2$, in which $n_0 < 1$, and so we only need fewer controls than total number of nodes, and
\item the unmanageable regime ``$\gamma \le \gamma_c$'', in which  $n_0 = 1$, where each node must be controlled explicitly.
\end{itemize}

\paragraph{Why linear dynamics ?}
As explained in \cite{Liu2011}, the choice of linear dynamics over a more general nonlinear dynamics can be justified as follows:
\begin{itemize}
  \item  Conclusions drawn from linear dynamics can be
extended to nonlinear systems.
\item If the controllability matrix of the linearized system
has full rank at all points, then it is sufficient for most
systems to say that the actual nonlinear system is
controllable (i.e. small signal model).
\end{itemize}

\paragraph{A note on stability.}
For stability in the LTI model \eqref{eq:lti} and hence in the constrained path integral \eqref{eq:hj} below, a sufficient (and necessary ?) condition is that all eigenvalues of the $A$ have negative real-part. One way to impose this is to add self-loops with a small negative weight.
\paragraph{Meta brain.}

\subsection{Optimal control: an LQR feedback controller}
We propose to use a linear quadratic regulator (LQR) for the feedback controller. Thus, consider the time-varying \text{value function} $V: [0, \tau] \times \mathbb R^n \rightarrow \mathbb R$, defined by the following Hamilton-Jacobi cost-to-go
\begin{equation}
  \begin{split}
    &V(t, \z) := \min_{\u} \frac{1}{2}\x_{\text{fin}}^T\Q_{\text{fin}}\x_{\text{fin}} + \int_{0}^\tau\left(\frac{1}{2}\x(t)^T\Q\x(t) + \frac{1}{2}\u(t)^T\R\u(t)\right)dt,\\
    &\text{subject to } \x(t) = \z, \dot{\x}(t) = \A\x(t) + \B\u(t),
  \end{split}
  \label{eq:hj}
\end{equation}
where the matrices $\Q$ and $\R$, precised by design considerations and subject to meta-optimization, are restricted to be positive semi-definite and positive definite, respectively. A classical calculation (reminiscent of the \textit{Pontryagin minimization principle}) reveals that the optimal feedback control in \eqref{eq:hj} is given by
\begin{equation}
  \u(t) = -\K(t)\x(t),
\end{equation}
where
\begin{equation}
  \K(t) := \R^{-1}\B^T\P(t)\text{ (Kalman gain matrix)}
\end{equation}
and the time-varying positive semi-definite matrix $\P(t)$ solves following differential Riccati equation (DRE)
\begin{equation}
\begin{split}
&-\dot{\P}(t) = \A^T\P(t) + \P(t)\A - \P(t)\B\R^{-1}\B^T\P(t) + \Q\\
&\text{subject to }\P(\tau) = \Q_{\text{fin}}.
\end{split}
\end{equation}
\subsection{Minimal control energy controller}
\subsection{Related works}
The model proposed in \cite{betzel2016} can be cast in the form \eqref{eq:hj}, with the particular choice $\Q = \textbf{I} = \rho^{-1}\R$ and $\Q_{\text{fin}} = \textbf{0}$, but with a rather ad-hoc handle on stability issues and choice driver nodes...

\paragraph{Data.}

\section{Experimental Results}
\paragraph{Serial versus parallel structure discovery and classification.}


\section{Discussion and Conclusion}



\paragraph{Acknowledgment}
% {\small The research leading to these results has received funding from the
% European Union Seventh Framework Programme (FP7/2007-2013)
% under grant agreement no. 604102 (Human Brain Project).
% Further support was received from
% the German National Academic Foundation (D.B.).
% }

\small
\bibliographystyle{splncs03}
\bibliography{nips_refs}

\end{document}
