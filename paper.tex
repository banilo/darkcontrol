\documentclass{article} % For LaTeX2e
\usepackage{nips14submit_e,times}
\usepackage{hyperref}
\usepackage{url}
\usepackage{amsmath,amsfonts,amsthm}
\usepackage{bbm}
\usepackage{algorithm,algorithmic}
\usepackage{graphicx}
\usepackage{bm}
\usepackage{bbm}
\usepackage[titletoc]{appendix}
\usepackage{wrapfig}
\usepackage{afterpage}
\usepackage{amssymb}
\usepackage{booktabs}
\usepackage{ulem}
\usepackage{multirow}

\def\B#1{\bm{#1}}
%\def\B#1{\mathbf{#1}}
\def\trans{\mathsf{T}}

%\renewcommand{\labelitemi}{--}

\newtheorem{theorem}{Theorem} \newtheorem{lemma}[theorem]{Lemma}
\newtheorem{proposition}[theorem]{Proposition}
\newtheorem{corollary}[theorem]{Corollary}
\newtheorem{definition}[theorem]{Definition}
\newtheorem{remark}{Remark}

%%%%%%%%%%%%%%%%%%%%%%%%%%%%%%%%%%%%%%%%%%%%%%%%%%%%%%%%%%%%%%%%%%%%%%%%%%%%%%%

\title{The Default Mode of Brain Function\\ Viewed By Control Theory}

\newcommand{\fix}{\marginpar{FIX}}
\newcommand{\new}{\marginpar{NEW}}
\DeclareMathOperator{\proj}{proj}
\DeclareMathOperator{\softmax}{softmax}
\DeclareMathOperator{\prox}{prox}
\DeclareMathOperator{\Prox}{Prox}
\DeclareMathOperator{\im}{im}

% macros from michael's .tex
\DeclareMathOperator{\dist}{dist} % The distance.
\DeclareMathOperator{\argmin}{argmin}
\DeclareMathOperator{\argmax}{argmax}
\DeclareMathOperator{\Id}{Id}
\DeclareMathOperator{\abs}{abs}
\newcommand{\R}{\mathbb{R}}
\newcommand{\N}{\mathbb{N}}
\newtheorem{thm}{Theorem}[section]
\newtheorem{prop}[thm]{Proposition}
\newtheorem{lem}[thm]{Lemma}
\newtheorem{cor}[thm]{Corollary}


\newcommand{\suggestadd}[1]{{\color{blue} #1}}
\newcommand{\suggestremove}[1]{{\color{red} \sout{#1}}}

% \nipsfinalcopy % Uncomment for camera-ready version
\nipsfinaltrue
%%%%%%%%%%%%%%%%%%%%%%%%%%%%%%%%%%%%%%%%%%%%%%%%%%%%%%%%%%%%%%%%%%%%%%%%%%%%%%%

\begin{document}

\author{Elvis Dohmatob, (Bert Kappen), Danilo Bzdok\\
  INRIA, Parietal team, Saclay, France\\
  CEA, Neurospin, Gif-sur-Yvette, France\\
  firstname.lastname@inria.fr}

\maketitle

\begin{abstract}
ABC
%

% OUR keywords
%\textbf{\\keywords}: curse of dimensionality; semi-supervised learning;
%fMRI; systems neuroscience

% official NIPS keywords
\textbf{\\keywords}: Systems Biology, Cognitive Science, Autonomous Learning Systems

\end{abstract}

\section{Introduction}
%
\paragraph{The Human Default Mode Network.}
\begin{itemize}
  \item includes brain regions with highest baseline energy consumption in humans
  \item most active in non-disturbed mind-wandering without external influence
  \item tends to deactivate when humans engage in goal-direct tasks
  \item fluctuations in the DMN are related to lapses during and performance in
  such externally structured/focused tasks
  \item includes brain regions that come online first when waking up from
  anesthesia
  \item its network nodes are late to myelinate, an indicator of cognitive sophistication
  \item includes the most advanced processes hierarchies and has
  no direct connections with sensory regions processing external input
  \item the DMN's conceivable key role in the continuous environmental
  tracking in a generative, integrative process might explain both its highest energy consumption in the brain and its intimate coupling with conscious awareness
  \item Indeed, patients with right IPL (=DMN) damage have particular difficulties
  with multistep actions (Hartmann et al. 2005)
  and with imagined action (Sirigu et al. 1996).
  \item Lesion of the hippocampus, feeding memory and spatial information, to the
  DMN impairs future and hypothetical thinking (cf. Hassabis PNAS)
  \item a lesion study on disturbed sleep (i.e., a state of mind
  independent of sensory stimulation but dependent on
  internally generated information) exclusively identified
  the dmPFC (Koenigs et al., 2010). Third, another VLSM study exclusively
  related the IFG and TPJ, both more strongly connected to the
  dmPFC in our study, to inner speech
  \item Bálint's syndrome: neurological disorder of conscious
  awareness resulting from damage in bilateral parietal cortex;
  "psychic paralysis of gaze", that is, the inability for internally
  motivated, targeted saccades hindering fixation of currently not
  attended features in the field of view (i.e., oculomotor apraxia).
  unable to bind various individual features of the visual
  environment into an integrated whole (i.e., simultanagnosia)
  as well as unable to navigate hand movement to a targeted
  object helped by vision (i.e., optic ataxia). 
  difficulty avoiding running into objects placed ahead;
  challenged in estimating distances as well as perceiving
  spatial depth, object sizes and object orientation
\end{itemize}

\paragraph{The Predictive Coding Framework.}
\begin{itemize}
  \item It is one of the most frequently cited hypotheses of default mode function
  \item Within this model the brain is conceptualized as a Bayesian machine
  \item Predictive coding is a framework that, in a hierarchical setting,
  is equivalent to empirical Bayesian inference
  \item External sensory input is feed-forward/bottom-up processed in the brain
  \item Their processing is compared against expected input
  \item The prediction error, i.e. the difference between sensory input
  and internal prediction, is computed at each level and passed to
  higher levels via forward connections
  \item in case of a mismatch, back-projections top-down modulate
  input processing
  \item ‘contextual integration’ for top-down modulation of sensorimotor
  processing by context-specific a-priori information.
  \item Closely related to Friston's free-energy principle: brain as inference engine
  biological systems, including brains, must minimize the long-term average of surprise;
  formulate perception as a constructive process based on internal or generative models
  \item A generative model of the world
\end{itemize}



\paragraph{Control Theory.}
\begin{itemize}
\item mathematically formalized dynamics systems with input, output and
feedback loops that produce error signals for adaptation
\item typically realized by (non-linear) differential equations
\item has been used to describe neural systems in the past
\item gradient update = learn/adapt system dynamics by error occurrence
\item related to notions of game theory
\end{itemize}



\section{Methods}
%
\paragraph{Data.}


\section{Experimental Results}
\paragraph{Serial versus parallel structure discovery and classification.}


\section{Discussion and Conclusion}



\paragraph{Acknowledgment}
% {\small The research leading to these results has received funding from the
% European Union Seventh Framework Programme (FP7/2007-2013)
% under grant agreement no. 604102 (Human Brain Project).
% Further support was received from
% the German National Academic Foundation (D.B.).
% }

\small
\bibliographystyle{splncs03}
\bibliography{nips_refs}

\end{document}
