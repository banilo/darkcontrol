\documentclass{article} % For LaTeX2e
\usepackage{nips14submit_e,times}
\usepackage{hyperref}
\usepackage{url}
\usepackage{amsmath,amsfonts,amsthm}
\usepackage{bbm}
\usepackage{algorithm,algorithmic}
\usepackage{graphicx}
\usepackage{bm}
\usepackage{bbm}
\usepackage[titletoc]{appendix}
\usepackage{wrapfig}
\usepackage{afterpage}
\usepackage{amssymb}
\usepackage{booktabs}
\usepackage{ulem}
\usepackage{multirow}

\def\B#1{\bm{#1}}
%\def\B#1{\mathbf{#1}}
\def\trans{\mathsf{T}}

%\renewcommand{\labelitemi}{--}

\newtheorem{theorem}{Theorem} \newtheorem{lemma}[theorem]{Lemma}
\newtheorem{proposition}[theorem]{Proposition}
\newtheorem{corollary}[theorem]{Corollary}
\newtheorem{definition}[theorem]{Definition}
\newtheorem{remark}{Remark}

%%%%%%%%%%%%%%%%%%%%%%%%%%%%%%%%%%%%%%%%%%%%%%%%%%%%%%%%%%%%%%%%%%%%%%%%%%%%%%%

\title{The Default Mode of Brain Function\\ Viewed By Control Theory}

\newcommand{\fix}{\marginpar{FIX}}
\newcommand{\new}{\marginpar{NEW}}
\DeclareMathOperator{\proj}{proj}
\DeclareMathOperator{\softmax}{softmax}
\DeclareMathOperator{\prox}{prox}
\DeclareMathOperator{\Prox}{Prox}
\DeclareMathOperator{\im}{im}

% macros from michael's .tex
\DeclareMathOperator{\dist}{dist} % The distance.
\DeclareMathOperator{\argmin}{argmin}
\DeclareMathOperator{\argmax}{argmax}
\DeclareMathOperator{\Id}{Id}
\DeclareMathOperator{\abs}{abs}
\newcommand{\R}{\mathbb{R}}
\newcommand{\N}{\mathbb{N}}
\newtheorem{thm}{Theorem}[section]
\newtheorem{prop}[thm]{Proposition}
\newtheorem{lem}[thm]{Lemma}
\newtheorem{cor}[thm]{Corollary}


\newcommand{\suggestadd}[1]{{\color{blue} #1}}
\newcommand{\suggestremove}[1]{{\color{red} \sout{#1}}}

% \nipsfinalcopy % Uncomment for camera-ready version
\nipsfinaltrue
%%%%%%%%%%%%%%%%%%%%%%%%%%%%%%%%%%%%%%%%%%%%%%%%%%%%%%%%%%%%%%%%%%%%%%%%%%%%%%%

\begin{document}

\author{Elvis Dohmatob, (Bert Kappen), Danilo Bzdok\\
  INRIA, Parietal team, Saclay, France\\
  CEA, Neurospin, Gif-sur-Yvette, France\\
  firstname.lastname@inria.fr}

\maketitle

\begin{abstract}
ABC
%

% OUR keywords
%\textbf{\\keywords}: curse of dimensionality; semi-supervised learning;
%fMRI; systems neuroscience

% official NIPS keywords
\textbf{\\keywords}: Systems Biology, Cognitive Science, Autonomous Learning Systems

\end{abstract}

\section{Introduction}
%
abc



\section{Methods}
%
\paragraph{Data.}


\section{Experimental Results}
\paragraph{Serial versus parallel structure discovery and classification.}


\section{Discussion and Conclusion}



\paragraph{Acknowledgment}
{\small The research leading to these results has received funding from the
European Union Seventh Framework Programme (FP7/2007-2013)
under grant agreement no. 604102 (Human Brain Project).
Data were provided by the Human Connectome Project.
Further support was received from
the German National Academic Foundation (D.B.).
}

\small
\bibliographystyle{splncs03}
\bibliography{nips_refs}

\end{document}
